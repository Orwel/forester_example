\documentclass[11pt]{article}
\usepackage[utf8]{inputenc}
\usepackage[francais]{babel}
\usepackage{graphicx}

\usepackage{glossaries}
\makeglossaries
\newglossaryentry{electrolyte}{name=electrolyte,
description={solution able to conduct electric current}}

\newglossaryentry{oesophagus}{name=\oe sophagus,
description={canal from mouth to stomach},
plural=\oe sophagi}

\newglossaryentry{oesophaguss}{name=\oe sophagus,
description={canal from mouth to stomach},
plural=\oe sophagi,sort=oesophagus}

\newglossaryentry{oesophagusss}{name=�sophagus,
description={canal from mouth to stomach},
plural=�sophagi}

\begin{document}
\title{Functionnality of Forester}
\author{Cédric Vernou\\
Programmeur émérite}

\maketitle
\tableofcontents
\newpage

Voici un court example mettant en scéne la génération de la table des matières, de la bibliographie, d'un index et d'un glossaire\cite{ARTMCCDDS}.
Enjoy \gls{electrolyte}.

\section {Introduction}

In my school project, I have worked on Forester. Presentely, Forester don't have documentation and I have exchanged a lot of email with authors of Forester. This document is a synthesis of this explanation and my experimentations.
\\

Forester is an experimental tool for checking manipulation of dynamic data
structures using *\gls{FA}*. The tool can be loaded into *GCC* as a *plug-in*.  This
way you can easily analyse C code sources, using the existing build system,
without manually preprocessing them first. However, the analysis itself is not yet mature enough to be able to verify ordinary programs.
\section {Forester}

Forester is an experimental tool for checking manipulation of dynamic data
structures using *\gls{FA}*



\printglossaries

\bibliographystyle{plain}
\bibliography{biblio}
\addcontentsline{toc}{chapter}{Bibliographie}

\end{document}