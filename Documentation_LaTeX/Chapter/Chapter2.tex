\section {Use Forester}

	\subsection{To use}

Forester is plugin GCC, you can use this command to load plugin in GCC and check code source :

gcc -fplugin=libfa.so
\\

But Forester is provided with some scripts for ease of use. In the first place, you need to configurate some variables in console with this script "./fa\_build/register-paths.sh". You can use this command :

. ./fa\_build/register-paths.sh
\\

This script add paths in your terminal toward other scripts to execute easily Forester. Now, you can use Forester scripts to execute GCC and to load Forester. The main script is :

fagcc main.c
\\

	\subsection{Some remarks}
	
This scripts are really easy to use, but they have a flaw. You can only check one file at a time.
\\
In connection with this flaw the only file in parameter shall contain the main method. Else Forester returns you this error :

\lstset{language=C, numbers=left, frame=shadowbox}
\begin{lstlisting}
error: main() not found at global scope
\end{lstlisting}
\bigskip

A method to check your files: it is using an other forester.c file and include all files. In forester.c, write main method where you call all methods you want test.

\lstset{language=C, numbers=left, frame=shadowbox}
\begin{lstlisting}
	#include "list.c"

	int main()
	{
		Node * list = initializeList(1);
		finalizeList(list);
		return 0;
	}
\end{lstlisting}
\bigskip